\documentclass[aspectratio=169]{beamer}
\usepackage{xeCJK}
\usepackage{fontspec}
\usepackage{graphicx}
\usepackage{listings}
\usepackage{xcolor}
\usepackage{indentfirst}
\usepackage{tikz}
\usepackage{amssymb}
\usepackage{amsthm}
\usepackage{amsmath}
\usepackage{tabularx}
\usepackage{hyperref}
\usepackage{ulem}
\usepackage{version}
\usepackage{thmtools}
\usepackage{qtree}
\usepackage{algpseudocode}
\usepackage{mathtools}
\usepackage{multicol}
\usepackage{forest}
\usepackage{minted}
\usepackage{colortbl}

\XeTeXlinebreaklocale "zh"
\XeTeXlinebreakskip = 0pt plus 1pt

\setCJKmainfont{Noto Sans CJK TC}
% \setmainfont{Noto Sans CJK TC}
\setmonofont{Ubuntu Mono}
\usetikzlibrary{arrows,decorations.markings,decorations.pathreplacing}

\lstset{
    basicstyle=\ttfamily\normalsize\color{black},
    commentstyle=\color{black!50},
    keywordstyle=\color{white!0!blue},
    stringstyle=\color{black!50!green},
    showspaces=false,
    showstringspaces=false,
    showtabs=false,
    tabsize=4,
    captionpos=b,
    breaklines=true,
    breakatwhitespace=false,
    escapeinside={\%*}{*)},
    morekeywords={*}
}
\forestset{
  default preamble = {
    for tree = {
      edge = {-},
      circle,
      minimum size=2.5em,
      inner sep=0pt,
      draw,
      math content,
      tier/.wrap pgfmath arg={tier #1}{level()},
      anchor=center,
      s sep = 2em
    }
  }
}
\newcommand{\inc}[2][cpp]{\mintinline{#1}|#2|}

\title{Complexity}
\author{上課補充 by PixelCat} % \\Credit by XXX
\date{}

\usetheme{sprout}

\setlength\parskip{12pt}
\makeatletter
\newcommand{\@minipagerestore}{\setlength{\parskip}{12pt}}
\makeatother

\hypersetup{colorlinks=true, linkcolor=SproutLinkColor, urlcolor=SproutURLColor}

\begin{document}

{\setbeamertemplate{background}
    {\includegraphics[width=\paperwidth,height=\paperheight,keepaspectratio]{background_title.png}}
    \begin{frame}
        \titlepage
    \end{frame}
}

\section{課程影片}

\begin{frame}{課程影片}
    Q \& A?
\end{frame}


\section{數學的複雜度定義}

\begin{frame}{極限?}
  課程影片說複雜度是用極限去算的

  \only<1> {
    \begin{definition}[極限]
      $\lim_{n \to \infty} f(n) = L$,定義為:
      對任意的 $\epsilon > 0$,存在某個 $N$,使得對於所有 $n > N$,$|f(n) - L| < \epsilon$。
    \end{definition}
  }

  \only<2> {
    \begin{definition}[極限(白話版本)]
      $\lim_{n \to \infty} f(n) = L$,表示在 $n$ 超大的時候 $f(n)$ 會很接近很接近某個確切的值
    \end{definition}

    有時候極限會不存在:$f(n)$ 可以無限制的變大、$f(n)$ 沒有定義、等等情況
  }
\end{frame}

\begin{frame}{漸近複雜度}
  複雜度的定義和極限長很像(但不太一樣)

  \only<1> {
    \begin{definition}[Big-O 複雜度]
      $f(n) = O(g(n))$,定義為:
      存在常數 $N, c > 0$,使得對於所有 $n \ge N$,$0 \le f(n) \le cg(n)$。
    \end{definition}
  }

  \only<2> {
    \begin{definition}[Big-O 複雜度]
      $f(n) = O(g(n))$,表示在 $n$ 超大的時候,$f(n)$ 和 $g(n)$ 的常數倍比大小,$f(n)$ 會一直輸下去。
    \end{definition}
  }
\end{frame}

\begin{frame}{漸近複雜度}
  \begin{corollary}[推論(錯誤)]
    $f(n) = O(g(n))$,代表:$\lim_{n \to \infty} \frac{f(n)}{g(n)} \neq \infty$
  \end{corollary}

  這是錯的!
\end{frame}

\begin{frame}{漸近複雜度}
  \begin{corollary}[推論(正確)]
    $\lim_{n \to \infty} \frac{f(n)}{g(n)} \neq \infty$,代表:$f(n) = O(g(n))$
  \end{corollary}

  兩者有微妙的小差異
\end{frame}

\begin{frame}{漸近複雜度}
  複雜度不在乎你的函數具體是多少,只想比較兩個函數在遙遠的未來誰會比較大

  我們不在乎(沒辦法在乎)演算法具體要跑多久,只想知道兩個演算法在超大規模下誰需要比較少算力(比較快)
\end{frame}

\begin{frame}{漸近複雜度}
  \begin{center}
    \bgroup
      \def\arraystretch{1.3}
      \setlength\tabcolsep{1em}
      \only<1> {
        \begin{tabular}[pos]{| c | c | c |}
          \hline
          small-O     & $f(n) = o(g(n))$      & $\dots f(n) < cg(n) \dots$                 \\\hline
          \rowcolor{lime!50!gray!30!white}
          big-O       & $f(n) = O(g(n))$      & $\dots f(n) \le cg(n) \dots$               \\\hline
          big-theta   & $f(n) = \Theta(g(n))$ & $\dots c_1g(n) \le f(n) \le c_2g(n) \dots$ \\\hline
          big-omega   & $f(n) = \Omega(g(n))$ & $\dots cg(n) \le f(n) \dots$               \\\hline
          small-omega & $f(n) = \omega(g(n))$ & $\dots cg(n) < f(n) \dots$                 \\\hline
        \end{tabular}
      }
      \only<2> {
        \begin{tabular}[pos]{| c | c | l |}
          \hline
          small-O     & $f(n) = o(g(n))$      & $g(n)$ 是 $f(n)$ 的\textbf{上界}(嚴格大於) \\\hline
          \rowcolor{lime!50!gray!30!white}
          big-O       & $f(n) = O(g(n))$      & $g(n)$ 是 $f(n)$ 的\textbf{上界}(可以一樣) \\\hline
          big-theta   & $f(n) = \Theta(g(n))$ & $f(n)$ 跟 $g(n)$ 長\textbf{一樣快}          \\\hline
          big-omega   & $f(n) = \Omega(g(n))$ & $g(n)$ 是 $f(n)$ 的\textbf{下界}(可以一樣) \\\hline
          small-omega & $f(n) = \omega(g(n))$ & $g(n)$ 是 $f(n)$ 的\textbf{下界}(嚴格小於) \\\hline
        \end{tabular}
      }
    \egroup
  \end{center}
\end{frame}


\section{常見的複雜度與 NP-completeness}

\begin{frame}{比一比}
  誰比較大?

  \begin{center}
    \bgroup
      \def\arraystretch{1.3}
      \setlength\tabcolsep{1em}
      \begin{tabular}[h]{c c c}
        \hline
        $3n^2 + n + 20$ & vs & $100n$ \\ \hline
        $n ^ {100}$ & vs & $2 ^ n$ \\ \hline
        $n ^ 2$ & vs & $10 n \log n$ \\ \hline
        $100^n$ & vs & $n!$ \\ \hline
        $30 \times 2 ^ n$ & vs & $3 ^ n$ \\ \hline
        $100n$ & vs & $200n$ \\ \hline
      \end{tabular}
    \egroup
  \end{center}

  (credit:古代投影片 by Chin-Huang Lin)
\end{frame}

\begin{frame}{比一比}
  誰比較大?

  \begin{center}
    \bgroup
      \def\arraystretch{1.3}
      \setlength\tabcolsep{1em}
      \begin{tabular}[h]{c c c c c}
        \hline
        (複雜度比較高) & $3n^2 + n + 20$ & vs & $100n$ \\ \hline
        & $n ^ {100}$ & vs & $2 ^ n$ & (複雜度比較高) \\ \hline
        (複雜度比較高) & $n ^ 2$ & vs & $10 n \log n$ & \\ \hline
        & $100^n$ & vs & $n!$ & (複雜度比較高) \\ \hline
        & $30 \times 2 ^ n$ & vs & $3 ^ n$ & (複雜度比較高) \\ \hline
        (複雜度相同) & $100n$ & vs & $200n$ & (複雜度相同) \\ \hline
      \end{tabular}
    \egroup
  \end{center}

  (credit:古代投影片 by Chin-Huang Lin)
\end{frame}

\begin{frame}{複雜度的階級}
  誰比較大?

  \begin{center}
    \bgroup
      \def\arraystretch{1.3}
      \setlength\tabcolsep{1em}
      \begin{tabular}[h]{c c c}
        \hline
        $n ^ 2$ & vs & $2 ^ n$ \\ \hline
        \only<2-> {$n ^ {10}$ & vs & $2 ^ n$ \\ \hline }
        \only<3-> {$n ^ {10}$ & vs & $1.1 ^ n$ \\ \hline }
        \only<4-> {$n ^ {1000000000}$ & vs & $1.0000000001 ^ n$ \\ \hline }
      \end{tabular}
    \egroup
  \end{center}

  \only<5> {
    \textbf{多項式}時間的演算法跟\textbf{指數}時間的演算法相比,複雜度總是比較好
  }
\end{frame}

\begin{frame}{複雜度的階級}
  誰比較大?

  \begin{center}
    \bgroup
      \def\arraystretch{1.3}
      \setlength\tabcolsep{1em}
      \begin{tabular}[h]{c c c}
        \hline
        $n ^ 2$ & vs & $\log n$ \\ \hline
        $n ^ 2$ & vs & $\log ^ {10} n$ \\ \hline
        $n ^ {0.1}$ & vs & $\log ^ {10} n$ \\ \hline
        $n ^ {1.0000000001}$ & vs & $\log ^ {1000000000} n$ \\ \hline
      \end{tabular}
    \egroup
  \end{center}

  \only<2> {
    \textbf{多項式}時間的演算法跟\textbf{對數}時間的演算法相比,複雜度總是比較差
  }
\end{frame}

\begin{frame}{一個問題有多難?}
  相比於指數時間的演算法,多項式時間是巨大的進步

  有些問題可以輕鬆設計出多項式時間的演算法\\
  有些問題怎麼努力想,就是只想得到指數時間的演算法
\end{frame}

\begin{frame}{NP-completeness}
  決定性問題:只能回答 YES/NO 的問題

  根據「問題有多難在多項式時間解決」,我們把所有決定性問題歸類:

  \begin{itemize}
    \item P 問題:可以在多項式時間內解決
    \item NP 問題:可以在多項式時間內\textbf{驗證一組 YES 的解確實是對的}
    \item NP-hard 問題
    \begin{itemize}
      \item 本身未必是 NP 問題
      \item 只要多項式時間做出 NP-hard 問題,就可以多項式時間做出所有 NP 問題
    \end{itemize}
    \item NP-complete 問題:同時是 NP 問題和 NP-hard 問題
  \end{itemize}
\end{frame}

\begin{frame}{NP-completeness}
  NP 問題能夠在多項式時間解決嗎?

  知道答案的話不要告訴我,去發論文你就變世界偉人

  目前的普遍信念是 P $\neq$ NP\\
  也就是 NP 好難好難,難到沒辦法在多項式時間內解決
\end{frame}

\begin{frame}{NP-completeness}
  「知道問題不可做」和「知道問題要怎麼做」一樣重要

  情境一:題目怎麼看起來很像 NP-complete 問題?\\
     $\longrightarrow$ 可能看錯題目了、漏看條件了、出題者完蛋了

  情境二:我怎麼不小心做出 NP-complete 問題了?\\
     $\longrightarrow$ 想法出錯了
\end{frame}


\section{分析複雜度}

\begin{frame}{分析複雜度}
  假設某些\textbf{基本操作}需要的時間都差不多

  \begin{enumerate}
    \item 計算演算法需要做幾次\textbf{基本操作}
    \item 留下複雜度最大的那一項
  \end{enumerate}

  計算複雜度相當仰賴 case by case 討論,不只是數迴圈!
\end{frame}

\begin{frame}[fragile]{分析複雜度:數迴圈(一)}
  \begin{minted}{cpp}
    #define rep(i, n) for(int i = 0; i < n; i++)

    void mult(int n, int a[N][N], int b[N][N], int c[N][N]) {
      rep(i, n) rep(j, n) {
        c[i][j] = 0;
      }
      rep(i, n) rep(j, n) rep(k, n) {
        tmp[i][j] += (a[i][k] * b[k][j]);
      }
      rep(i, n) rep(j, n) {
        c[i][j] = tmp[i][j] % MOD;
      }
    }
  \end{minted}

  樸實的矩陣乘法
\end{frame}

\begin{frame}{分析複雜度:數迴圈(一)}
  \begin{itemize}
    \item $N ^ 3 + 2 N ^ 2$ 次賦值
    \item $N ^ 3$ 次加法
    \item $N ^ 3$ 次乘法
    \item $N ^ 2$ 次除法(模運算)
    \item ???次陣列取值、指標和位址計算\dots
  \end{itemize}

  確切不知道,大概是 $? \times N^3 + ? \times N^2 + \dots$ 次基本操作

  \only<2> {
    複雜度告訴你,$N$ 很大的時候常數和複雜度小的項沒什麼大影響

    複雜度就是妥妥的 $O(N^3)$
  }
\end{frame}

\begin{frame}{分析複雜度:數迴圈(一)}
  一層迴圈跑很多次\\
  迴圈套起來迭代次數會乘起來

  最多有幾層迴圈,複雜度就是幾次方(?!)
\end{frame}

\begin{frame}[fragile]{分析複雜度:數迴圈(二)}
  \begin{minted}{cpp}
    void f(int n) {
      int ans = 0;
      for(int i = 0; i < n; i++) {
        for(int j = 0; j < (1 << n); j++) {
          ans += i * j;
        }
      }
    }
  \end{minted}
\end{frame}

\begin{frame}[fragile]{分析複雜度:數迴圈(二)}
  \begin{minted}{cpp}
    int f(int n) {
      int ans = 0;
      for(int i = 0; i < n; i++) {  // 0 ... (n - 1)
        for(int j = 0; j < (1 << n); j++) {  // 0 ... (2^n - 1)
          ans += i * j;
        }
      }
      return ans;
    }
  \end{minted}

  時間複雜度:$O(n2^n)$
\end{frame}

\begin{frame}[fragile]{分析複雜度:數迴圈(三)}
  \begin{minted}{cpp}
    int g() {
      int ans = 0;
      for(int i = 0; i < 100; i++) {
        ans += i;
      }
      return ans;
    }
  \end{minted}

  \only<2> {
    雖然有點不甘願,但是 $O(100) = O(1)$ 確實是常數時間
  }
\end{frame}

\begin{frame}[fragile]{分析複雜度:數迴圈(四)}
  \begin{minted}{cpp}
    int my_lower_bound(int n, int key, int arr[]) {
      int lo = -1, hi = n;
      while(hi - lo > 1) {
        int mi = (hi + lo) / 2;
        if(arr[mi] >= key) hi = mi;
        else lo = mi;
      }
      return hi;
    }
  \end{minted}

  \only<2> {
    \inc{(hi - lo)} 一開始是 $n + 1$,每次都被砍一半,砍個 $\log N$ 次之後變 $1$ 退出迴圈

    二分搜尋,時間複雜度 $O(\log N)$
  }
\end{frame}

\begin{frame}[fragile]{分析複雜度:被藏起來的複雜度}
  \begin{minted}{cpp}
    std::sort(a + 1, a + n + 1);
  \end{minted}

  沒有迴圈,總共 $O(1)$(??)

  \only<2> {
    呼叫別的函數當然需要時間,\href{https://en.cppreference.com/w/}{cppreference} 之類的通常會告訴你各個內建函數的時間複雜度
  }
\end{frame}

\begin{frame}[fragile]{分析複雜度:遞迴函數}
  \begin{minted}{cpp}
    int gcd(int a, int b) {
      if(b == 0) return a;
      return gcd(b, a % b);
    }
  \end{minted}
  
  \only<2> {
    輾轉相除的時間複雜度是多少?
  }
  \only<2> {
    時間複雜度 $O(\log \min(a, b))$

    為什麼?!
  }
\end{frame}

\begin{frame}[fragile]{分析複雜度:遞迴函數}
  遞迴函數比較難搞,在這裡先略過
\end{frame}

\begin{frame}[fragile]{分析複雜度:均攤分析}
  \begin{minted}{cpp}
    for(int idx = 0; idx < n; idx++) {
      while(stk.size() && value[stk.top()] > value[idx]) {
        ans[stk.top()] = idx;
        stk.pop();
      }
      stk.push(idx);
    }
  \end{minted}

  \only<1> {
    上週教過的單調堆疊
  }
  \only<2> {
    \inc{for} 迴圈執行 $N$ 次\\
    \inc{while} 迴圈每一輪最多執行 $N$ 次(stack 最多裝 $N$ 個元素)

    總複雜度是 $O(N^2)$,真的那麼糟糕嗎
  }
  \only<3> {
    認真聽上週課程的你知道,不會有那麼多元素讓你 \inc{pop},從頭到尾 \inc{while} 迴圈總共最多跑 $N$ 次。時間複雜度 $O(N)$

    均攤分析「偶爾會跑很慢,但是不可能每次都跑很慢,平均起來還是跑很快」
  }
\end{frame}

\begin{frame}{分析複雜度}
  複雜度分析不單純是數迴圈、還需要豐富的經驗和數學和數學和數學
\end{frame}

\begin{frame}{分析複雜度}
  算完複雜度之後呢?

  \begin{itemize}
    \item 把題目給的變數範圍代進去,看看會不會超時
    \begin{itemize}
      \item 我的電腦可以一秒跑 $4 \times 10^9$ 次加法
      \item 綜合考量其他因素,代入複雜度後在 $10^7 \sim 10^8$ 通常算合理不超時範圍
    \end{itemize}
    \item 有沒有複雜度差、但夠快而且好寫的作法?
    \item 超時了,優化演算法的哪個地方可以改進複雜度?
    \begin{itemize}
      \item 例:少用一層迴圈?
    \end{itemize}
  \end{itemize}
\end{frame}


\section{複雜度之外的現實因素}

\begin{frame}{「常數」}
  複雜度的計算會忽略常數

  在線上評測系統,你不只要考慮演算法的複雜度,還要把他實做出來

  \begin{itemize}
    \item $N$ 次加法和 $2N$ 次加法,誰比較快?
    \item $N$ 次加法和 $N$ 次除法,誰比較快?
    \item ...?
  \end{itemize}
\end{frame}

\begin{frame}[fragile]{實驗一}
  \begin{minted}{cpp}
    const int MAXN = 100'000'000;
    int a[MAXN + 10];  // is assigned random value

    for(int i = 1; i <= MAXN; i++) ans = ans ^ a[i];
    // 0.021 s

    for(int i = 1; i <= MAXN; i++) ans = ans + a[i];
    // 0.022 s
    
    for(int i = 1; i <= MAXN; i++) ans = ans * a[i];
    // 0.065 s
    
    for(int i = 1; i <= MAXN; i++) ans = (ans * a[i]) % MOD;
    // 0.292 s
  \end{minted}

  不同運算需要的時間不一樣
\end{frame}

\begin{frame}[fragile]{實驗二}
  \begin{minted}{cpp}
    const int MAXN = 100'000'000;
    int a[MAXN + 10];    // is assigned random value in [0, 2^16)
    int ord[MAXN + 10];  // is assigned 1 ... MAXN

    for(int i = 1; i <= MAXN; i++) ans += a[ord[i]];
    // 0.035 s
    
    shuffle(ord + 1, ord + MAXN + 1);
    for(int i = 1; i <= MAXN; i++) ans += a[ord[i]];
    // 0.758 s
  \end{minted}

  「cache miss」
\end{frame}

\begin{frame}[fragile]{實驗三}
  \begin{minted}{cpp}
    const int MAXN = 10'000;
    int a[MAXN + 10][MAXN + 10];  // is assigned random value in [0, 2^16)

    for(int i = 1; i <= MAXN; i++)
      for(int j = 1; j <= MAXN; j++)
        ans += a[i][j];
    // 0.085 s
    
    for(int j = 1; j <= MAXN; j++)
      for(int i = 1; i <= MAXN; i++)
        ans += a[i][j];
    // 1.741 s
  \end{minted}

  也是 cache miss
\end{frame}

\begin{frame}[fragile]{實驗四}
  \begin{minted}{cpp}
    const int MAXN = 100'000'000;
    int a[MAXN + 10];  // is assigned random value in [0, 2^16)

    for(int i = 1; i <= MAXN; i++) ans = ans + a[i];
    // g++ main.cpp -O0
    // 0.166 s
    
    for(int i = 1; i <= MAXN; i++) ans = ans + a[i];
    // g++ main.cpp -O4
    // 0.045 s
  \end{minted}

  編譯器的邪惡優化
\end{frame}

\begin{frame}{常數優化}
  除了演算法以外,還有無數種因素會影響你的程式跑多久

  「常數優化」的目標是透過各種(邪惡的)手段,寫出演算法相同但執行更快的程式碼
\end{frame}

\begin{frame}{常數優化}
  但是這些手段通常優化效果有限

  「複雜度壓一個 $N$,$N = 1000$」和「常數優化讓程式變快 $20\%$」哪一個比較賺?

  設計更好的演算法更加重要!
\end{frame}


\end{document}
